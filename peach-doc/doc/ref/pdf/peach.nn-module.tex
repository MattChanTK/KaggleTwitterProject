%
% API Documentation for Peach - Computational Intelligence for Python
% Package peach.nn
%
% Generated by epydoc 3.0.1
% [Sun Jul 31 17:00:40 2011]
%

%%%%%%%%%%%%%%%%%%%%%%%%%%%%%%%%%%%%%%%%%%%%%%%%%%%%%%%%%%%%%%%%%%%%%%%%%%%
%%                          Module Description                           %%
%%%%%%%%%%%%%%%%%%%%%%%%%%%%%%%%%%%%%%%%%%%%%%%%%%%%%%%%%%%%%%%%%%%%%%%%%%%

    \index{peach \textit{(package)}!peach.nn \textit{(package)}|(}
\section{Package peach.nn}

    \label{peach:nn}

This package implements support for neural networks. Consult:
%
\begin{quote}
%
\begin{description}
\item[{base}] \leavevmode 
Basic definitions of the objects used with neural networks;

\item[{af}] \leavevmode 
A list of activation functions for use with neurons and a base class to
implement different activation functions;

\item[{lrule}] \leavevmode 
Learning rules;

\item[{nnet}] \leavevmode 
Implementation of different classes of neural networks;

\item[{mem}] \leavevmode 
Associative memories and Hopfield model;

\item[{kmeans}] \leavevmode 
K-Means implementation for use with Radial Basis Networks;

\item[{rbfn}] \leavevmode 
Radial Basis Function Networks;

\end{description}

\end{quote}

%%%%%%%%%%%%%%%%%%%%%%%%%%%%%%%%%%%%%%%%%%%%%%%%%%%%%%%%%%%%%%%%%%%%%%%%%%%
%%                                Modules                                %%
%%%%%%%%%%%%%%%%%%%%%%%%%%%%%%%%%%%%%%%%%%%%%%%%%%%%%%%%%%%%%%%%%%%%%%%%%%%

\subsection{Modules}

\begin{itemize}
\setlength{\parskip}{0ex}
\item \textbf{af}: 
Base activation functions and base class


  \textit{(Section \ref{peach:nn:af}, p.~\pageref{peach:nn:af})}

\item \textbf{base}: 
Basic definitions for layers of neurons.


  \textit{(Section \ref{peach:nn:base}, p.~\pageref{peach:nn:base})}

\item \textbf{kmeans}: 
K-Means clustering algorithm


  \textit{(Section \ref{peach:nn:kmeans}, p.~\pageref{peach:nn:kmeans})}

\item \textbf{lrules}: 
Learning rules for neural networks and base classes for custom learning.


  \textit{(Section \ref{peach:nn:lrules}, p.~\pageref{peach:nn:lrules})}

\item \textbf{mem}: 
Associative memories and Hopfield network model.


  \textit{(Section \ref{peach:nn:mem}, p.~\pageref{peach:nn:mem})}

\item \textbf{nnet}: 
Basic topologies of neural networks.


  \textit{(Section \ref{peach:nn:nnet}, p.~\pageref{peach:nn:nnet})}

\item \textbf{rbfn}: 
Radial Basis Function Networks


  \textit{(Section \ref{peach:nn:rbfn}, p.~\pageref{peach:nn:rbfn})}

\end{itemize}

    \index{peach \textit{(package)}!peach.nn \textit{(package)}|)}
