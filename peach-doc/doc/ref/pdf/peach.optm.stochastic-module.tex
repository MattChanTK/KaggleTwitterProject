%
% API Documentation for Peach - Computational Intelligence for Python
% Module peach.optm.stochastic
%
% Generated by epydoc 3.0.1
% [Sun Jul 31 17:00:42 2011]
%

%%%%%%%%%%%%%%%%%%%%%%%%%%%%%%%%%%%%%%%%%%%%%%%%%%%%%%%%%%%%%%%%%%%%%%%%%%%
%%                          Module Description                           %%
%%%%%%%%%%%%%%%%%%%%%%%%%%%%%%%%%%%%%%%%%%%%%%%%%%%%%%%%%%%%%%%%%%%%%%%%%%%

    \index{peach \textit{(package)}!peach.optm \textit{(package)}!peach.optm.stochastic \textit{(module)}|(}
\section{Module peach.optm.stochastic}

    \label{peach:optm:stochastic}

%%%%%%%%%%%%%%%%%%%%%%%%%%%%%%%%%%%%%%%%%%%%%%%%%%%%%%%%%%%%%%%%%%%%%%%%%%%
%%                               Variables                               %%
%%%%%%%%%%%%%%%%%%%%%%%%%%%%%%%%%%%%%%%%%%%%%%%%%%%%%%%%%%%%%%%%%%%%%%%%%%%

  \subsection{Variables}

    \vspace{-1cm}
\hspace{\varindent}\begin{longtable}{|p{\varnamewidth}|p{\vardescrwidth}|l}
\cline{1-2}
\cline{1-2} \centering \textbf{Name} & \centering \textbf{Description}& \\
\cline{1-2}
\endhead\cline{1-2}\multicolumn{3}{r}{\small\textit{continued on next page}}\\\endfoot\cline{1-2}
\endlastfoot\raggedright \_\-\_\-d\-o\-c\-\_\-\_\- & \raggedright \textbf{Value:} 
{\tt \texttt{...}}&\\
\cline{1-2}
\end{longtable}


%%%%%%%%%%%%%%%%%%%%%%%%%%%%%%%%%%%%%%%%%%%%%%%%%%%%%%%%%%%%%%%%%%%%%%%%%%%
%%                           Class Description                           %%
%%%%%%%%%%%%%%%%%%%%%%%%%%%%%%%%%%%%%%%%%%%%%%%%%%%%%%%%%%%%%%%%%%%%%%%%%%%

    \index{peach \textit{(package)}!peach.optm \textit{(package)}!peach.optm.stochastic \textit{(module)}!peach.optm.stochastic.CrossEntropy \textit{(class)}|(}
\subsection{Class CrossEntropy}

    \label{peach:optm:stochastic:CrossEntropy}
\begin{tabular}{cccccc}
% Line for peach.optm.Optimizer, linespec=[False]
\multicolumn{2}{r}{\settowidth{\BCL}{peach.optm.Optimizer}\multirow{2}{\BCL}{peach.optm.Optimizer}}
&&
  \\\cline{3-3}
  &&\multicolumn{1}{c|}{}
&&
  \\
&&\multicolumn{2}{l}{\textbf{peach.optm.stochastic.CrossEntropy}}
\end{tabular}


Multidimensional search based on cross-entropy technique.

In cross-entropy, a set of N possible solutions is randomly generated at
each interaction. To converge the solutions, the best M solutions are
selected and its statistics are calculated. A new set of solutions are
randomly generated from these statistics.

%%%%%%%%%%%%%%%%%%%%%%%%%%%%%%%%%%%%%%%%%%%%%%%%%%%%%%%%%%%%%%%%%%%%%%%%%%%
%%                                Methods                                %%
%%%%%%%%%%%%%%%%%%%%%%%%%%%%%%%%%%%%%%%%%%%%%%%%%%%%%%%%%%%%%%%%%%%%%%%%%%%

  \subsubsection{Methods}

    \label{peach:optm:stochastic:CrossEntropy:__init__}
    \index{peach \textit{(package)}!peach.optm \textit{(package)}!peach.optm.stochastic \textit{(module)}!peach.optm.stochastic.CrossEntropy \textit{(class)}!peach.optm.stochastic.CrossEntropy.\_\_init\_\_ \textit{(method)}}

    \vspace{0.5ex}

\hspace{.8\funcindent}\begin{boxedminipage}{\funcwidth}

    \raggedright \textbf{\_\_init\_\_}(\textit{self}, \textit{f}, \textit{M}={\tt 30}, \textit{N}={\tt 60}, \textit{emax}={\tt 1e-8}, \textit{imax}={\tt 1000})

    \vspace{-1.5ex}

    \rule{\textwidth}{0.5\fboxrule}
\setlength{\parskip}{2ex}

Initializes the optimizer.

To create an optimizer of this type, instantiate the class with the
parameters given below:
\setlength{\parskip}{1ex}
      \textbf{Parameters}
      \vspace{-1ex}

      \begin{quote}
        \begin{Ventry}{xxxx}

          \item[f]


A multivariable function to be optimized. The function should have
only one parameter, a multidimensional line-vector, and return the
function value, a scalar.
          \item[M]


Size of the solution set used to calculate the statistics to
generate the next set of solutions
          \item[N]


Total size of the solution set.
          \item[emax]


Maximum allowed error. The algorithm stops as soon as the error is
below this level. The error is absolute.
          \item[imax]


Maximum number of iterations, the algorithm stops as soon this
number of iterations are executed, no matter what the error is at
the moment.
        \end{Ventry}

      \end{quote}

    \end{boxedminipage}

    \label{peach:optm:stochastic:CrossEntropy:step}
    \index{peach \textit{(package)}!peach.optm \textit{(package)}!peach.optm.stochastic \textit{(module)}!peach.optm.stochastic.CrossEntropy \textit{(class)}!peach.optm.stochastic.CrossEntropy.step \textit{(method)}}

    \vspace{0.5ex}

\hspace{.8\funcindent}\begin{boxedminipage}{\funcwidth}

    \raggedright \textbf{step}(\textit{self})

    \vspace{-1.5ex}

    \rule{\textwidth}{0.5\fboxrule}
\setlength{\parskip}{2ex}

One step of the search (\emph{NOT IMPLEMENTED YET})

In this method, the solution set is searched for the M best solutions.
Mean and variance of these solutions is calculated, and these values are
used to randomly generate, from a gaussian distribution, a set of N new
solutions.
\setlength{\parskip}{1ex}
    \end{boxedminipage}

    \index{peach \textit{(package)}!peach.optm \textit{(package)}!peach.optm.stochastic \textit{(module)}!peach.optm.stochastic.CrossEntropy \textit{(class)}|)}
    \index{peach \textit{(package)}!peach.optm \textit{(package)}!peach.optm.stochastic \textit{(module)}|)}
