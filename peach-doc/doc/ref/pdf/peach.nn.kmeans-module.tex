%
% API Documentation for Peach - Computational Intelligence for Python
% Module peach.nn.kmeans
%
% Generated by epydoc 3.0.1
% [Sun Jul 31 17:00:40 2011]
%

%%%%%%%%%%%%%%%%%%%%%%%%%%%%%%%%%%%%%%%%%%%%%%%%%%%%%%%%%%%%%%%%%%%%%%%%%%%
%%                          Module Description                           %%
%%%%%%%%%%%%%%%%%%%%%%%%%%%%%%%%%%%%%%%%%%%%%%%%%%%%%%%%%%%%%%%%%%%%%%%%%%%

    \index{peach \textit{(package)}!peach.nn \textit{(package)}!peach.nn.kmeans \textit{(module)}|(}
\section{Module peach.nn.kmeans}

    \label{peach:nn:kmeans}

K-Means clustering algorithm

This sub-package implements the K-Means clustering algorithm. This algorithm,
given a set of points, finds a set of vectors that best represents a partition
for these points. These vectors represent the center of a cloud of points that
are nearest to them.

This algorithm is one that can be used with radial basis function (RBF) networks
to find the centers of the RBFs. Usually, training a RBFN in two passes -{}- first
positioning them, and then computing their variance.

%%%%%%%%%%%%%%%%%%%%%%%%%%%%%%%%%%%%%%%%%%%%%%%%%%%%%%%%%%%%%%%%%%%%%%%%%%%
%%                               Functions                               %%
%%%%%%%%%%%%%%%%%%%%%%%%%%%%%%%%%%%%%%%%%%%%%%%%%%%%%%%%%%%%%%%%%%%%%%%%%%%

  \subsection{Functions}

    \label{peach:nn:kmeans:ClassByDistance}
    \index{peach \textit{(package)}!peach.nn \textit{(package)}!peach.nn.kmeans \textit{(module)}!peach.nn.kmeans.ClassByDistance \textit{(function)}}

    \vspace{0.5ex}

\hspace{.8\funcindent}\begin{boxedminipage}{\funcwidth}

    \raggedright \textbf{ClassByDistance}(\textit{xs}, \textit{c})

    \vspace{-1.5ex}

    \rule{\textwidth}{0.5\fboxrule}
\setlength{\parskip}{2ex}

Given a set of points and a list of centers, classify the points according
to their euclidian distance to the centers.
\setlength{\parskip}{1ex}
      \textbf{Parameters}
      \vspace{-1ex}

      \begin{quote}
        \begin{Ventry}{xx}

          \item[xs]


Set of points to be classified. They must be given as a list or array of
one-dimensional vectors, one per line.
          \item[c]


Set of centers. Must also be given as a lista or array of
one-dimensional vectors, one per line.
        \end{Ventry}

      \end{quote}

      \textbf{Return Value}
    \vspace{-1ex}

      \begin{quote}

A list of index of the classification. The indices are the position of the
cluster in the given parameters \texttt{c}.
      \end{quote}

    \end{boxedminipage}

    \label{peach:nn:kmeans:ClusterByMean}
    \index{peach \textit{(package)}!peach.nn \textit{(package)}!peach.nn.kmeans \textit{(module)}!peach.nn.kmeans.ClusterByMean \textit{(function)}}

    \vspace{0.5ex}

\hspace{.8\funcindent}\begin{boxedminipage}{\funcwidth}

    \raggedright \textbf{ClusterByMean}(\textit{x})

    \vspace{-1.5ex}

    \rule{\textwidth}{0.5\fboxrule}
\setlength{\parskip}{2ex}

This function computes the center of a cluster by averaging the vectors in
the input set by simply averaging each component.
\setlength{\parskip}{1ex}
      \textbf{Parameters}
      \vspace{-1ex}

      \begin{quote}
        \begin{Ventry}{x}

          \item[x]


Set of points to be clustered. They must be given in the form of a list
or array of one-dimensional points.
        \end{Ventry}

      \end{quote}

      \textbf{Return Value}
    \vspace{-1ex}

      \begin{quote}

A one-dimensional array representing the center of the cluster.
      \end{quote}

    \end{boxedminipage}


%%%%%%%%%%%%%%%%%%%%%%%%%%%%%%%%%%%%%%%%%%%%%%%%%%%%%%%%%%%%%%%%%%%%%%%%%%%
%%                               Variables                               %%
%%%%%%%%%%%%%%%%%%%%%%%%%%%%%%%%%%%%%%%%%%%%%%%%%%%%%%%%%%%%%%%%%%%%%%%%%%%

  \subsection{Variables}

    \vspace{-1cm}
\hspace{\varindent}\begin{longtable}{|p{\varnamewidth}|p{\vardescrwidth}|l}
\cline{1-2}
\cline{1-2} \centering \textbf{Name} & \centering \textbf{Description}& \\
\cline{1-2}
\endhead\cline{1-2}\multicolumn{3}{r}{\small\textit{continued on next page}}\\\endfoot\cline{1-2}
\endlastfoot\raggedright \_\-\_\-d\-o\-c\-\_\-\_\- & \raggedright \textbf{Value:} 
{\tt \texttt{...}}&\\
\cline{1-2}
\raggedright \_\-\_\-p\-a\-c\-k\-a\-g\-e\-\_\-\_\- & \raggedright \textbf{Value:} 
{\tt \texttt{'}\texttt{peach.nn}\texttt{'}}&\\
\cline{1-2}
\end{longtable}


%%%%%%%%%%%%%%%%%%%%%%%%%%%%%%%%%%%%%%%%%%%%%%%%%%%%%%%%%%%%%%%%%%%%%%%%%%%
%%                           Class Description                           %%
%%%%%%%%%%%%%%%%%%%%%%%%%%%%%%%%%%%%%%%%%%%%%%%%%%%%%%%%%%%%%%%%%%%%%%%%%%%

    \index{peach \textit{(package)}!peach.nn \textit{(package)}!peach.nn.kmeans \textit{(module)}!peach.nn.kmeans.KMeans \textit{(class)}|(}
\subsection{Class KMeans}

    \label{peach:nn:kmeans:KMeans}
\begin{tabular}{cccccc}
% Line for object, linespec=[False]
\multicolumn{2}{r}{\settowidth{\BCL}{object}\multirow{2}{\BCL}{object}}
&&
  \\\cline{3-3}
  &&\multicolumn{1}{c|}{}
&&
  \\
&&\multicolumn{2}{l}{\textbf{peach.nn.kmeans.KMeans}}
\end{tabular}


K-Means clustering algorithm

This class implements the known and very used K-Means clustering algorithm.
In this algorithm, the centers of the clusters are selected randomly. The
points on the training set are classified in accord to their closeness to
the cluster centers. This changes the positions of the centers, which
changes the classification of the points. This iteration is repeated until
no changes occur.

Traditional K-Means implementations classify the points in the training set
according to the euclidian distance to the centers, and centers are computed
as the average of the points associated to it. This is the default behaviour
of this implementation, but it is configurable. Please, read below for more
detail.

%%%%%%%%%%%%%%%%%%%%%%%%%%%%%%%%%%%%%%%%%%%%%%%%%%%%%%%%%%%%%%%%%%%%%%%%%%%
%%                                Methods                                %%
%%%%%%%%%%%%%%%%%%%%%%%%%%%%%%%%%%%%%%%%%%%%%%%%%%%%%%%%%%%%%%%%%%%%%%%%%%%

  \subsubsection{Methods}

    \vspace{0.5ex}

\hspace{.8\funcindent}\begin{boxedminipage}{\funcwidth}

    \raggedright \textbf{\_\_init\_\_}(\textit{self}, \textit{training\_set}, \textit{nclusters}, \textit{classifier}={\tt {\textless}function ClassByDistance at 0x901e1ec{\textgreater}}, \textit{clusterer}={\tt {\textless}function ClusterByMean at 0x901e374{\textgreater}})

    \vspace{-1.5ex}

    \rule{\textwidth}{0.5\fboxrule}
\setlength{\parskip}{2ex}

Initializes the algorithm.
\setlength{\parskip}{1ex}
      \textbf{Parameters}
      \vspace{-1ex}

      \begin{quote}
        \begin{Ventry}{xxxxxxxxxxxx}

          \item[training\_set]


A list or array of vectors containing the data to be classified.
Each of the vectors in this list \emph{must} have the same dimension, or
the algorithm won't behave correctly. Notice that each vector can be
given as a tuple -{}- internally, everything is converted to arrays.
          \item[nclusters]


The number of clusters to be found. This must be, of course, bigger
than 1. These represent the number of centers found once the
algorithm terminates.
          \item[classifier]


A function that classifies each of the points in the training set.
This function receives the training set and a list of centers, and
classify each of the points according to the given metric. Please,
look at the documentation on these functions for more information.
Its default value is %
\raisebox{1em}{\hypertarget{id2}{}}\hyperlink{id1}{\textbf{\color{red}``}}ClassByDistance` , which uses euclidian
distance as metric.
          \item[clusterer]


A function that computes the center of the cluster, given a set of
points. This function receives a list of points and returns the
vector representing the cluster. For more information, look at the
documentation for these functions. Its default value is
\texttt{ClusterByMean}, in which the cluster is represented by the mean
value of the vectors.
        \end{Ventry}

      \end{quote}

      Overrides: object.\_\_init\_\_

    \end{boxedminipage}

    \label{peach:nn:kmeans:KMeans:step}
    \index{peach \textit{(package)}!peach.nn \textit{(package)}!peach.nn.kmeans \textit{(module)}!peach.nn.kmeans.KMeans \textit{(class)}!peach.nn.kmeans.KMeans.step \textit{(method)}}

    \vspace{0.5ex}

\hspace{.8\funcindent}\begin{boxedminipage}{\funcwidth}

    \raggedright \textbf{step}(\textit{self})

    \vspace{-1.5ex}

    \rule{\textwidth}{0.5\fboxrule}
\setlength{\parskip}{2ex}

This method runs one step of the algorithm. It might be useful to track
the changes in the parameters.
\setlength{\parskip}{1ex}
      \textbf{Return Value}
    \vspace{-1ex}

      \begin{quote}

The computed centers for this iteration.
      \end{quote}

    \end{boxedminipage}

    \label{peach:nn:kmeans:KMeans:__call__}
    \index{peach \textit{(package)}!peach.nn \textit{(package)}!peach.nn.kmeans \textit{(module)}!peach.nn.kmeans.KMeans \textit{(class)}!peach.nn.kmeans.KMeans.\_\_call\_\_ \textit{(method)}}

    \vspace{0.5ex}

\hspace{.8\funcindent}\begin{boxedminipage}{\funcwidth}

    \raggedright \textbf{\_\_call\_\_}(\textit{self}, \textit{imax}={\tt 20})

    \vspace{-1.5ex}

    \rule{\textwidth}{0.5\fboxrule}
\setlength{\parskip}{2ex}

The \texttt{\_\_call\_\_} interface is used to run the algorithm until
convergence is found.
\setlength{\parskip}{1ex}
      \textbf{Parameters}
      \vspace{-1ex}

      \begin{quote}
        \begin{Ventry}{xxxx}

          \item[imax]


Specifies the maximum number of iterations admitted in the execution
of the algorithm. It defaults to 20.
        \end{Ventry}

      \end{quote}

      \textbf{Return Value}
    \vspace{-1ex}

      \begin{quote}

An array containing, at each line, the vectors representing the
centers of the clustered regions.
      \end{quote}

    \end{boxedminipage}


\large{\textbf{\textit{Inherited from object}}}

\begin{quote}
\_\_delattr\_\_(), \_\_format\_\_(), \_\_getattribute\_\_(), \_\_hash\_\_(), \_\_new\_\_(), \_\_reduce\_\_(), \_\_reduce\_ex\_\_(), \_\_repr\_\_(), \_\_setattr\_\_(), \_\_sizeof\_\_(), \_\_str\_\_(), \_\_subclasshook\_\_()
\end{quote}

%%%%%%%%%%%%%%%%%%%%%%%%%%%%%%%%%%%%%%%%%%%%%%%%%%%%%%%%%%%%%%%%%%%%%%%%%%%
%%                              Properties                               %%
%%%%%%%%%%%%%%%%%%%%%%%%%%%%%%%%%%%%%%%%%%%%%%%%%%%%%%%%%%%%%%%%%%%%%%%%%%%

  \subsubsection{Properties}

    \vspace{-1cm}
\hspace{\varindent}\begin{longtable}{|p{\varnamewidth}|p{\vardescrwidth}|l}
\cline{1-2}
\cline{1-2} \centering \textbf{Name} & \centering \textbf{Description}& \\
\cline{1-2}
\endhead\cline{1-2}\multicolumn{3}{r}{\small\textit{continued on next page}}\\\endfoot\cline{1-2}
\endlastfoot\raggedright c\- & &\\
\cline{1-2}
\multicolumn{2}{|l|}{\textit{Inherited from object}}\\
\multicolumn{2}{|p{\varwidth}|}{\raggedright \_\_class\_\_}\\
\cline{1-2}
\end{longtable}

    \index{peach \textit{(package)}!peach.nn \textit{(package)}!peach.nn.kmeans \textit{(module)}!peach.nn.kmeans.KMeans \textit{(class)}|)}
    \index{peach \textit{(package)}!peach.nn \textit{(package)}!peach.nn.kmeans \textit{(module)}|)}
